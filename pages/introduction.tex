\section{Введение}
\par{C++ -- язык программирования созданный в 1983 датским ученым в области компьютерных наук Бьйорном Страуструпом. Изначально его идеалогия заключалась в том, что требуется дополнить язык C классами для удобства работы в парадигме ООП. В последствии он развивался мнодествами сообществами программистов, что привело к сложности взаимодействия между этими сообществами. В связи с этим в 1998 году было создана группа стандартизации C++ -- WG21. С помощью этой группы(так же именнуемой коммитетом стандартизации или просто коммитетом) был создан первый стандарт -- C++98, описывающий базовый функционал языка.}
\par{Дальнейшие стандарты начали стабильно выходить с 2011 года с периодичностью три года(с тех пор стандарты называются по формату C++\{две последние цифры года\}) и привнесли множество новых фичей<поменять>, таких как: модель памяти, лямбда-функции, модель потоков, реализация классических алгоритмов и многое другое.}
\par{В этой работе мы сфокусировались на совершенствовании стандартной библиотеки поддержки многопоточности. В частности планируется разработать функционал выставления потоку приоритета так, что бы не приходилось создавать отдельный код для всевозможных API, а так же дополнить функцию std::thread::sleep\_for до вида в котором она существует в std::jthread(то есть добавить stop\_token).}
\par{Помимо этого были проанализированы алгоритмы бинарного поиска и сортировки в реализации стандартной библиотеки и найдены способы оптимизировать их.}
\par{В разделе \ref{literature_review} осуществляется краткий обзор источников используемых по всей работе. Затем начинаются разделы который описывают решение тех или иных задач: оптимизация бинарного поиска \ref{binary_search} и сортировки \ref{sort}; создание функционала потоков \ref{thread_priority} и токена сна \ref{sleep_token}.}
